\documentclass{article}
\usepackage{amsmath}
\usepackage{graphics}
\usepackage{color}
\definecolor{DarkGreen}{rgb}{0, .5, 0}
\definecolor{DarkBlue}{rgb}{0, 0, .5}
\definecolor{DarkRed}{rgb}{.5, 0, 0}
\definecolor{LightGray}{rgb}{.95, .95, .95}
\definecolor{White}{rgb}{1.0,1.0,1.0}
\definecolor{darkblue}{rgb}{0,0,0.9}
\definecolor{darkred}{rgb}{0.8,0,0}
\definecolor{darkgreen}{rgb}{0.0,0.85,0}

\usepackage{listings}
\lstset{  
  language=[ISO]C++,                       % The default language
  backgroundcolor=\color{White},       % Set listing background
  keywordstyle=\color{DarkBlue}\bfseries,  % Set keyword style
  commentstyle=\color{DarkGreen}\itshape, % Set comment style
  stringstyle=\color{DarkRed},             % Set string constant style
  extendedchars=true                       % Allow extended characters
  breaklines=true,
  basewidth={0.5em,0.4em},
  fontadjust=true,
  linewidth=\textwidth,
  breakatwhitespace=true,
  lineskip=0ex, %  frame=single
}

\newcommand{\li}{\lstinline}
\newenvironment{lstlistings}
{\begin{lstlisting}[basicstyle=\sf]}
{\end{lstlisting}}
\newcommand{\cpp}[1]{\li! #1 !}
\begin{document}
\section{New release 0.2 December 2018}

\begin{itemize}
\item Added block diagonal preconditioner and preconditioned MinRes iterative methods. 
\item Bettered specification of data through file.
\item Now \cpp{dataPtr} is a global variable in the namespace \cpp{FVCode3D}. The data has been made public. This way the addition of features controlled by file is eased greatly. The datafile is loaded in the main and the variable \cpp{dataPtr} is accessible to all translation units that include \cpp{Data.hpp}.
\end{itemize}
\section{New release 0.1, October 2018}
This release has several enhancements. First of all permeabilities, porosity and fracture aperture  are given 
via the data file and they may be specified by zones. See the \li!data.txt! file in the examples.

Moreover, now it is possible to specify and load run time user-specific functions to describe both boundary conditions and source term.
The user defined functions have to be stored in a \emph{shared library} whose name is given to the program via the data file with the keyword
\li!functionLibrary=functionName!, where \li!functionname! is in the usual form for a shared library. In the examples the shared library is built automatically, by a suitable set the cmake files. But it can be also generated on the fly directly calling the compiler:

\begin{verbatim}
g++ -fPIC -shared -o libname.so -I<FVCode3Ddir> funsource.cpp
\end{verbatim}

You have to remember to indicate the directory at the top of the \li!src! hierarchy so that the required files are included. \li!FVCode3Ddir! is the directory under which \li!utility/! and \li!core/! reside.

A prototype for the file containing the functions is

\begin{lstlisting}
#include <FVCode3D/core/TypeDefinition.hpp>
#include <cmath>
FVCode3D::Func SourceTerm = [](FVCode3D::Point3D p)
{return 15.*( (
              (p.x()-0.3)*(p.x()-0.3) +
              (p.y()-0.3)*(p.y()-0.3) +
                (p.z()-0.8)*(p.z()-0.8)
              ) <=0.04
                        )
     -15.*( (
            (p.x()-0.5)*(p.x()-0.5) +
            (p.y()-0.5)*(p.y()-0.5) +
            (p.z()-0.3)*(p.z()-0.3)
             ) <=0.04);};
\end{lstlisting}

Note that you have to follow the syntax. Do not wrap the functions in a namespace. You may also use functors if you prefer. Note also that there are some in-built functions that you do not need to create: \li!fZero! and \li!fOne!. They return always zero and one.

See the data files available in the examples. They are heavily commented.

\section{Problems to be addressed}
\begin{enumerate}
\item The class \cpp{Point3D} has a major flaw! There is an overloading of \cpp{operator <} that contains a tolerance, and so it does not define
an ordering relation. I suggest to verify where the tolerance is needed and use a special comparison operator there. For instance by defining a
\begin{lstlisting}
struct CompareWithTolerance
{
  bool operator()(Point3D a, Point3D b);
\\ etc.
}
\end{lstlisting}
\item Properties are linked to zones, that however I have still to understand how they are given. Probably from the mesh data.
\end{enumerate}

\end{document}
%%% Local Variables:
%%% mode: latex
%%% TeX-master: t
%%% End:
